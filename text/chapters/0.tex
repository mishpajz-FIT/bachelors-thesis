% \chapter{Introduction}
% uncomment the following line to create an unnumbered chapter
\chapter*{Introduction}\addcontentsline{toc}{chapter}{Introduction}\markboth{Introduction}{Introduction}
\setcounter{page}{1}

% The following environment can be used as a mini-introduction for a chapter. Use that anyway it pleases you (or comment it out). It can contain, for instance, a summary of the chapter. Or, there can be a quotation.
\begin{chapterabstract}
Product configurators and their value.
\end{chapterabstract}

Over the past few decades, the rise of e-commerce has caused a shift in consumer expectations, resulting in an increased demand for individualized products. This gives rise to the need to shift focus towards mass customization, where products are customized according to individual preferences. To thrive in this sector, companies must modify their product offerings to meet the unique needs of users. This necessitates the existence of a system that enables customers to express their preferences and convert them into product configurations. \cite{Fulkerson2000}

The task of transforming user preferences into concrete designs is a difficult endeavor that can be hindered by a lack of effective communication between the customer's explanation of their desires and the business's comprehension. The use of online product configurators seemingly provides a solution to this issue by offering a user-friendly and visually appealing platform that allows customers to customize products according to their specifications, improves the customer experience by increasing engagement and interactivity, and helps bridge the gap between customer expectations and the end product. These tools have become an integral part of successful personalization strategies. \cite{Franke2003}

The involvement of consumers in the customization process leads to a stronger bond with the product compared to standard off-the-shelf products. This aspect of mass customization makes it an appealing and compelling strategy for businesses to implement. \cite{Schreier2006} However, when implementing such a system, it is crucial to ensure that the customization process is pleasurable for the customer. Research has shown that the enjoyment experienced during the customization process also affects the perceived value of the final product, highlighting the importance of good implementation. \cite{Franke2010}

The introduction of modern technologies such as WebGL or  \acrfull{ar} has expanded the potential of online configurators. These advances enable these toolkits to become more powerful and visually illustrative tools that provide a higher level of interactivity and realism than what was previously accessible. \cite{Cozzi2015}

%---------------------------------------------------------------
\section*{Objective of This Thesis}
%---------------------------------------------------------------

The primary objective of this thesis is to design and implement an application (toolkit) for the 3D online configuration of modular products. The toolkit aims to be product-agnostic, adaptable, and customizable, usable by various businesses, enabling their customers to customize their modular products interactively. The focus is on ensuring that the toolkit is not only flexible in accommodating various specific needs, but also straightforward for businesses to maintain after deployment, emphasizing lightweight infrastructure requirements. 

To accomplish this main objective, an accompanying analysis of the characteristics found in current product configurators is required, as well as an examination of comparable solutions currently available to businesses.

%---------------------------------------------------------------
\section*{Structure of This Thesis}
%---------------------------------------------------------------

This thesis is divided into six chapters.

\begin{description}
\item[Chapter 1] The initial chapter entails an examination of existing solutions and an investigation into the functionalities that should be incorporated into this particular application.

\item[Chapter 2] The second chapter discusses the design of the application and the technologies chosen.

\item[Chapter 3] The third chapter is devoted to implementation.

\item[Chapter 4] Chapter four focuses on the example deployment of the implemented application in a particular business. In addition, it discusses the resulting changes in the business processes of a chosen example business.

\item[Chapter 5] In the fifth chapter, the tests used in the development of the application are described.

\item[Chapter 6] Finally, the last chapter summarizes the results achieved and suggests possible directions for future development.
\end{description}