\chapter{Conclusion}

The objective of this bachelor's thesis was to create an application for online~3D~configuration of modular products.

To accomplish this goal, an analysis of existing solutions was performed, examining the perspective of both customers and the businesses that operate such tools. Based on this, requirements for the solution implemented in this thesis were created, such that the created application provides businesses with an innovative solution for product configuration.

Consequently, a responsive web application was designed to support the~configuration of a variety of modular products in a 3D environment. The~configurator features open navigation, allowing users to customize modular products by adding or removing specified components at fixed points, and to customize their properties such as the color of the materials. Additionally, advanced features such as collision detection were included to ensure that configurations are realistic and feasible. Further processing of the user-created configurations has been made available by offering an integration using an \noborderacrshort{api} call to other systems or by presenting an inquiry form.

From a business perspective, the subsequent goals were for the application to be flexible, lightweight, and easy to maintain, targeting smaller companies in need of a cost-effective solution. The developed tool is a front-end-only solution, set up by static files. To allow customers to configure their products, businesses can deploy the developed solution directly on their webservers. The tool as a whole has been separated into two separate applications: the user-facing configurator and an administrator tool allowing the business to define their configurable products.

In the design chapter, technologies were chosen and wireframes of the user interfaces were crafted.

The implementation chapter detailed the data schemas and the solutions to several interesting challenges encountered during the implementation of the~designed application.

In the deployment chapter, the deployment process was described within the context of an example business, highlighting the potential changes to the~business processes enabled by this solution.

Furthermore, the testing performed during the development of the application was discussed, including unit testing, system testing, and usability testing. Usability testing was conducted near the end of the development cycle to validate the functionality and user experience of the application, and the most severe revealed usability issues were fixed.

Future improvements to this solution could include the creation of an~accompanying back-end solution, integrations with existing e-commerce platforms, or the implementation of a rule evaluation engine. Possible future development directions are discussed in detail in~\autoref{section:improvements}, and insights from usability testing for user experience improvements are also provided in~\autoref{section:insights}.

The developed solution is freely available for businesses to use, enabling them to introduce modular product configuration on their websites effectively.