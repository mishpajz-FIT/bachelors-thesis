\chapter{Conclusion}

The objective of this bachelor's thesis was to create an application for the 3D configuration of modular products.

To accomplish this goal, an analysis of existing solutions was performed, examining the perspective of both customers and the businesses operating such tools. Based on this, requirements for the solution implemented in this thesis were created, such that the created application provides businesses with an innovative solution for product configuration.

Consequently, a responsive web application was designed to support the configuration of a variety of modular products in a 3D environment. The configurator features open navigation, allowing users to customize modular products by adding or removing specified components at fixed points, and to customize their properties such as color. Additionally, advanced features such as collision detection were included to ensure that configurations are realistic and feasible. Further processing of the user-created configurations was made available by offering an integration using an API call to other systems or by presenting an inquiry form.

From a business perspective, the subsequent goals were for the  application to be flexible, lightweight, and easy to maintain, targeting smaller companies in need of a cost-effective solution. The developed tool is a front-end-only solution, set up by static files. To allow customers to configure their products, businesses can deploy the developed solution directly on their webservers. The tool as a whole has been separated into two separate applications: the user facing configurator and an administration tool allowing the business to define their configurable products.

In the design chapter, technologies were chosen and wireframes of the user interfaces were crafted.

The implementation chapter detailed the data schemas and the solutions to several interesting challenges encountered during implementation of the designed application.

In the deployment chapter, the deployment process was described within the context of a example business, highlighting the potential changes to business processes enabled by this solution.

Furthemore, the testing chapter detailed user testing of the application was conducted to validate functionality and user experience.

Further future improvements of this solution could include creation of accompanying back-end solution, integrations with existing e-Commerce platforms, or the implementation of a configuration rule evaluation engine. Possible future improvements were discussed in detail in \autoref{section:improvements}.

The developed solution is freely available for businesses to use, enabling them to introduce modular product configuration on their websites effectively.

\todo{Testing expansion}