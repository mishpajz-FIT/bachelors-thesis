\chapter{Additional Code Listings} 

\begin{listing}
\begin{minted}{typescript}
const renderer = new THREE.WebGLRenderer();
renderer.setSize(window.innerWidth, window.innerHeight);
document.body.appendChild(renderer.domElement);

const scene = new THREE.Scene();

const geometry = new THREE.BoxGeometry(5, 5, 5);
const material = new THREE.MeshBasicMaterial({color: 0xff0000});
const mesh = new THREE.Mesh(geometry, material);
scene.add(mesh);

const camera = new THREE.PerspectiveCamera(
  75,
  window.innerWidth / window.innerHeight,
  0.1,
  1000
);
camera.position.set(10, 10, 10);
camera.lookAt(mesh.position);

renderer.render(scene, camera);
\end{minted}
\caption{Creating and displaying a 3D red cube with Three.js}
\label{listing:threejs}
\end{listing}

\begin{listing}
\begin{minted}{typescript}
const Component = () => {
    return (
        <Canvas camera={{position: [10, 10, 10]}}>
            <mesh>
                <meshBasicMaterial color="red" />
                <boxGeometry args={[5, 5, 5]} />
            </mesh>
        </Canvas>
    )
}
\end{minted}
\caption{Creating a 3D red cube as a React component with R3F}
\label{lisiting:r3f}
\end{listing}

\begin{listing}
\begin{minted}{typescript}
import { z } from 'zod';

const customSchema = z.number();

type CustomType = z.infer<typeof customSchema>;
\end{minted}
\caption{Conversion from Zod schema to TypeScript type}
\label{listing:zod}
\end{listing}